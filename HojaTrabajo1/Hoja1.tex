\documentclass[12pt,a4paper]{article}
\usepackage[latin1]{inputenc}
\usepackage[spanish]{babel}
\usepackage{amsmath}
\usepackage{amsfonts}
\usepackage{amssymb}
\usepackage{graphicx}
\usepackage[left=2cm,right=2cm,top=2cm,bottom=2cm]{geometry}
\author{Javier Jose Alvarez Flores, 20171074, alvarez171074@unis.edu.gt}
\title{Hoja de trabajo No.1}
\date{25 de enero del 2018}

\begin{document}
\title{Hoja de trabajo 1}
\maketitle{}

\section{Que hacer}
\begin{itemize}

\item Tipo de dato: string  -Nombre: propiedad que contiene el nombre del que hacer.
\item Tipo de dato: datetime - Fecha: propiedad que contiene la fecha en la que se realizara el que hacer.
\item Tipo de dato: string - Lugar: propiedad que describe el lugar en donde ha de llevarse a cabo el que hacer.
\item Tipo de dato: string - Descripcion: propiedad que descrique en que consiste la tarea.
\item Tipo de dato: string - Materiales: propiedad que describe los materiales necesarios para poder llevarse a cabo la tarea.
\item Tipo de dato: string - Prioridad: propiedad que contiene el nivel de importancia del que hacer a realizar.

\end{itemize}

\section{Que haceres}
\begin{enumerate}
\item Agregar: funcion que se encarga de agregar un nuevo elemento a la lista.
\item Obtener por el index: funcion que busca un elemento de la lista a partir del índice que se le envie como parametro.
\item Eliminar: funcion encargada de remover un elemento de la lista.
\item Modificar: funcion encargada de actualizar las propiedades de un elemento de la lista.
\item Longitud: propiedad numerica que indica la cantidad de elementos que posee la lista.
\end{enumerate}


\end{document}