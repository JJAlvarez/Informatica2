\documentclass[12pt,a4paper]{article}
\usepackage[latin1]{inputenc}
\usepackage[spanish]{babel}
\usepackage{amsmath}
\usepackage{amsfonts}
\usepackage{amssymb}
\usepackage{graphicx}
\usepackage[left=2cm,right=2cm,top=2cm,bottom=2cm]{geometry}
\author{Javier Jose Alvarez Flores, 20171074, alvarez171074@unis.edu.gt}
\title{Hoja de trabajo No.2}
\date{1 de febrero del 2018}

\begin{document}
\title{Hoja de trabajo 2}
\maketitle{}

\section{Respuesta a la pregunta (Confusion)}
\begin{itemize}

\item El resultado se debe a que cuando se ejecuta el metodo CompletarQueHacer se completa una tarea asignada a las dos personas, al hacer esto y debido que la tarea asignada a la persona uno y a la persona dos apuntan al mismo puntero en memoria, ambas se completan, causando que ambas personas esten disponibles.

\end{itemize}



\end{document}