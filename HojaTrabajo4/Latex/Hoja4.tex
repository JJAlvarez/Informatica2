\documentclass{article}
\usepackage[utf8]{inputenc}
\usepackage{fancyhdr} % Required for custom headers
\usepackage{lastpage} % Required to determine the last page for the footer
\usepackage{extramarks} % Required for headers and footers
\usepackage[usenames,dvipsnames]{color} % Required for custom colors
\usepackage{graphicx} % Required to insert images
\usepackage{listings} % Required for insertion of code
\usepackage{courier} % Required for the courier font
\usepackage{multirow}
\usepackage{hyperref}

% Margins
\topmargin=-0.45in
\evensidemargin=0in
\oddsidemargin=0in
\textwidth=6.5in
\textheight=9.0in
\headsep=0.25in

\linespread{1.1} % Line spacing

\definecolor{MyDarkGreen}{rgb}{0.0,0.4,0.0} % This is the color used for comments
\lstloadlanguages{c} % Load Perl syntax for listings, for a list of other languages supported see: ftp://ftp.tex.ac.uk/tex-archive/macros/latex/contrib/listings/listings.pdf
\lstset{language=[sharp]c, % Use Perl in this example
        frame=single, % Single frame around code
        basicstyle=\small\ttfamily, % Use small true type font
        keywordstyle=[1]\color{Blue}\bf, % Perl functions bold and blue
        keywordstyle=[2]\color{Purple}, % Perl function arguments purple
        keywordstyle=[3]\color{Blue}\underbar, % Custom functions underlined and blue
        identifierstyle=, % Nothing special about identifiers                                         
        commentstyle=\usefont{T1}{pcr}{m}{sl}\color{MyDarkGreen}\small, % Comments small dark green courier font
        stringstyle=\color{Purple}, % Strings are purple
        showstringspaces=false, % Don't put marks in string spaces
        tabsize=5, % 5 spaces per tab
        %
        % Put standard Perl functions not included in the default language here
        morekeywords={rand},
        %
        % Put Perl function parameters here
        morekeywords=[2]{on, off, interp},
        %
        % Put user defined functions here
        morekeywords=[3]{test},
       	%
        morecomment=[l][\color{Blue}]{...}, % Line continuation (...) like blue comment
        numbers=left, % Line numbers on left
        firstnumber=1, % Line numbers start with line 1
        numberstyle=\tiny\color{Blue}, % Line numbers are blue and small
        stepnumber=5 % Line numbers go in steps of 5
}

\newcommand{\horrule}[1]{\rule{\linewidth}{#1}}

\newcommand{\perlscript}[2]{
\begin{itemize}
\item[]\lstinputlisting[caption=#2,label=#1]{#1.cs}
\end{itemize}
}

\title{Hoja de trabajo #4}
\author{Javier Jose Alvarez Flores, 20171074, alvarez171074@unis.edu.gt}
\date{25 de febrero del 2018}

\begin{document}

\maketitle
\section{Explicacion}
Las ventajas que existen de utilizar metodos genericos a diferencia de no utilizarlos radican en varias partes, la primera de todas es que al no utilizar tipos genericos yo no puedo utilizar el mismo metodo (en este caso el metodo Head) para poder obtener el primer indice de un array de strings si el metodo solo recibe enteros, tendria que escribir otro metodo Head para poder hacer esta accion, y asi con cada tipo que me interese. 
\\\\
Esta problematica tiene solucion la cual es obtener un arreglo de tipo object ya que de object heredan todos. Pero esto tiene un problema. Al hacer esto yo no puedo controlar que tipo de datos ha sido enviado ni que tipo de dato sera retornado, lo cual se puede ver en el ejemplo brindado, el compilador no presentara problema ya que tanto int como string heredad de object, y al momento de obtner la respuesta necesito castearla al tipo de dato de mi interes, lo caal lansaria una excepcion debido a que no puedo castear implicitamente un string a un entero, causando un error en tiempo de ejecucion. 
\\\\
Esto se soluciona con los metodos genericos, ya que con ellos puedo controlar el tipo de dato que se envia como parametro y tambien puedo restringir el retorno, tambien evitanto la necesidad del casteo de la respuesta ya que conozco el tipo que sera retornado.
\\\\
\perlscript{ejemplo}{}

\end{document}
